
% Default to the notebook output style

    


% Inherit from the specified cell style.




    
\documentclass[11pt]{article}

    
    
    \usepackage[T1]{fontenc}
    % Nicer default font (+ math font) than Computer Modern for most use cases
    \usepackage{mathpazo}
	\usepackage{float}
    % Basic figure setup, for now with no caption control since it's done
    % automatically by Pandoc (which extracts ![](path) syntax from Markdown).
    \usepackage{graphicx}
    % We will generate all images so they have a width \maxwidth. This means
    % that they will get their normal width if they fit onto the page, but
    % are scaled down if they would overflow the margins.
    \makeatletter
    \def\maxwidth{\ifdim\Gin@nat@width>\linewidth\linewidth
    \else\Gin@nat@width\fi}
    \makeatother
    \let\Oldincludegraphics\includegraphics
    % Set max figure width to be 80% of text width, for now hardcoded.
    \renewcommand{\includegraphics}[1]{\Oldincludegraphics[width=.8\maxwidth]{#1}}
    % Ensure that by default, figures have no caption (until we provide a
    % proper Figure object with a Caption API and a way to capture that
    % in the conversion process - todo).
    \usepackage{caption}
    \DeclareCaptionLabelFormat{nolabel}{}
    \captionsetup{labelformat=nolabel}

    \usepackage{adjustbox} % Used to constrain images to a maximum size 
    \usepackage{xcolor} % Allow colors to be defined
    \usepackage{enumerate} % Needed for markdown enumerations to work
    \usepackage{geometry} % Used to adjust the document margins
    \usepackage{amsmath} % Equations
    \usepackage{amssymb} % Equations
    \usepackage{textcomp} % defines textquotesingle
    % Hack from http://tex.stackexchange.com/a/47451/13684:
    \AtBeginDocument{%
        \def\PYZsq{\textquotesingle}% Upright quotes in Pygmentized code
    }
    \usepackage{upquote} % Upright quotes for verbatim code
    \usepackage{eurosym} % defines \euro
    \usepackage[mathletters]{ucs} % Extended unicode (utf-8) support
    \usepackage[utf8x]{inputenc} % Allow utf-8 characters in the tex document
    \usepackage{fancyvrb} % verbatim replacement that allows latex
    \usepackage{grffile} % extends the file name processing of package graphics 
                         % to support a larger range 
    % The hyperref package gives us a pdf with properly built
    % internal navigation ('pdf bookmarks' for the table of contents,
    % internal cross-reference links, web links for URLs, etc.)
    \usepackage{hyperref}
    \usepackage{longtable} % longtable support required by pandoc >1.10
    \usepackage{booktabs}  % table support for pandoc > 1.12.2
    \usepackage[inline]{enumitem} % IRkernel/repr support (it uses the enumerate* environment)
    \usepackage[normalem]{ulem} % ulem is needed to support strikethroughs (\sout)
                                % normalem makes italics be italics, not underlines
    \usepackage{mathrsfs}
    

    
    
    % Colors for the hyperref package
    \definecolor{urlcolor}{rgb}{0,.145,.698}
    \definecolor{linkcolor}{rgb}{.71,0.21,0.01}
    \definecolor{citecolor}{rgb}{.12,.54,.11}

    % ANSI colors
    \definecolor{ansi-black}{HTML}{3E424D}
    \definecolor{ansi-black-intense}{HTML}{282C36}
    \definecolor{ansi-red}{HTML}{E75C58}
    \definecolor{ansi-red-intense}{HTML}{B22B31}
    \definecolor{ansi-green}{HTML}{00A250}
    \definecolor{ansi-green-intense}{HTML}{007427}
    \definecolor{ansi-yellow}{HTML}{DDB62B}
    \definecolor{ansi-yellow-intense}{HTML}{B27D12}
    \definecolor{ansi-blue}{HTML}{208FFB}
    \definecolor{ansi-blue-intense}{HTML}{0065CA}
    \definecolor{ansi-magenta}{HTML}{D160C4}
    \definecolor{ansi-magenta-intense}{HTML}{A03196}
    \definecolor{ansi-cyan}{HTML}{60C6C8}
    \definecolor{ansi-cyan-intense}{HTML}{258F8F}
    \definecolor{ansi-white}{HTML}{C5C1B4}
    \definecolor{ansi-white-intense}{HTML}{A1A6B2}
    \definecolor{ansi-default-inverse-fg}{HTML}{FFFFFF}
    \definecolor{ansi-default-inverse-bg}{HTML}{000000}

    % commands and environments needed by pandoc snippets
    % extracted from the output of `pandoc -s`
    \providecommand{\tightlist}{%
      \setlength{\itemsep}{0pt}\setlength{\parskip}{0pt}}
    \DefineVerbatimEnvironment{Highlighting}{Verbatim}{commandchars=\\\{\}}
    % Add ',fontsize=\small' for more characters per line
    \newenvironment{Shaded}{}{}
    \newcommand{\KeywordTok}[1]{\textcolor[rgb]{0.00,0.44,0.13}{\textbf{{#1}}}}
    \newcommand{\DataTypeTok}[1]{\textcolor[rgb]{0.56,0.13,0.00}{{#1}}}
    \newcommand{\DecValTok}[1]{\textcolor[rgb]{0.25,0.63,0.44}{{#1}}}
    \newcommand{\BaseNTok}[1]{\textcolor[rgb]{0.25,0.63,0.44}{{#1}}}
    \newcommand{\FloatTok}[1]{\textcolor[rgb]{0.25,0.63,0.44}{{#1}}}
    \newcommand{\CharTok}[1]{\textcolor[rgb]{0.25,0.44,0.63}{{#1}}}
    \newcommand{\StringTok}[1]{\textcolor[rgb]{0.25,0.44,0.63}{{#1}}}
    \newcommand{\CommentTok}[1]{\textcolor[rgb]{0.38,0.63,0.69}{\textit{{#1}}}}
    \newcommand{\OtherTok}[1]{\textcolor[rgb]{0.00,0.44,0.13}{{#1}}}
    \newcommand{\AlertTok}[1]{\textcolor[rgb]{1.00,0.00,0.00}{\textbf{{#1}}}}
    \newcommand{\FunctionTok}[1]{\textcolor[rgb]{0.02,0.16,0.49}{{#1}}}
    \newcommand{\RegionMarkerTok}[1]{{#1}}
    \newcommand{\ErrorTok}[1]{\textcolor[rgb]{1.00,0.00,0.00}{\textbf{{#1}}}}
    \newcommand{\NormalTok}[1]{{#1}}
    
    % Additional commands for more recent versions of Pandoc
    \newcommand{\ConstantTok}[1]{\textcolor[rgb]{0.53,0.00,0.00}{{#1}}}
    \newcommand{\SpecialCharTok}[1]{\textcolor[rgb]{0.25,0.44,0.63}{{#1}}}
    \newcommand{\VerbatimStringTok}[1]{\textcolor[rgb]{0.25,0.44,0.63}{{#1}}}
    \newcommand{\SpecialStringTok}[1]{\textcolor[rgb]{0.73,0.40,0.53}{{#1}}}
    \newcommand{\ImportTok}[1]{{#1}}
    \newcommand{\DocumentationTok}[1]{\textcolor[rgb]{0.73,0.13,0.13}{\textit{{#1}}}}
    \newcommand{\AnnotationTok}[1]{\textcolor[rgb]{0.38,0.63,0.69}{\textbf{\textit{{#1}}}}}
    \newcommand{\CommentVarTok}[1]{\textcolor[rgb]{0.38,0.63,0.69}{\textbf{\textit{{#1}}}}}
    \newcommand{\VariableTok}[1]{\textcolor[rgb]{0.10,0.09,0.49}{{#1}}}
    \newcommand{\ControlFlowTok}[1]{\textcolor[rgb]{0.00,0.44,0.13}{\textbf{{#1}}}}
    \newcommand{\OperatorTok}[1]{\textcolor[rgb]{0.40,0.40,0.40}{{#1}}}
    \newcommand{\BuiltInTok}[1]{{#1}}
    \newcommand{\ExtensionTok}[1]{{#1}}
    \newcommand{\PreprocessorTok}[1]{\textcolor[rgb]{0.74,0.48,0.00}{{#1}}}
    \newcommand{\AttributeTok}[1]{\textcolor[rgb]{0.49,0.56,0.16}{{#1}}}
    \newcommand{\InformationTok}[1]{\textcolor[rgb]{0.38,0.63,0.69}{\textbf{\textit{{#1}}}}}
    \newcommand{\WarningTok}[1]{\textcolor[rgb]{0.38,0.63,0.69}{\textbf{\textit{{#1}}}}}
    
    
    % Define a nice break command that doesn't care if a line doesn't already
    % exist.
    \def\br{\hspace*{\fill} \\* }
    % Math Jax compatibility definitions
    \def\gt{>}
    \def\lt{<}
    \let\Oldtex\TeX
    \let\Oldlatex\LaTeX
    \renewcommand{\TeX}{\textrm{\Oldtex}}
    \renewcommand{\LaTeX}{\textrm{\Oldlatex}}
    % Document parameters
    % Document title
    \title{Lab 9}
    \author{Jiarong Ye}
    
    
    
    

    % Pygments definitions
    
\makeatletter
\def\PY@reset{\let\PY@it=\relax \let\PY@bf=\relax%
    \let\PY@ul=\relax \let\PY@tc=\relax%
    \let\PY@bc=\relax \let\PY@ff=\relax}
\def\PY@tok#1{\csname PY@tok@#1\endcsname}
\def\PY@toks#1+{\ifx\relax#1\empty\else%
    \PY@tok{#1}\expandafter\PY@toks\fi}
\def\PY@do#1{\PY@bc{\PY@tc{\PY@ul{%
    \PY@it{\PY@bf{\PY@ff{#1}}}}}}}
\def\PY#1#2{\PY@reset\PY@toks#1+\relax+\PY@do{#2}}

\expandafter\def\csname PY@tok@w\endcsname{\def\PY@tc##1{\textcolor[rgb]{0.73,0.73,0.73}{##1}}}
\expandafter\def\csname PY@tok@c\endcsname{\let\PY@it=\textit\def\PY@tc##1{\textcolor[rgb]{0.25,0.50,0.50}{##1}}}
\expandafter\def\csname PY@tok@cp\endcsname{\def\PY@tc##1{\textcolor[rgb]{0.74,0.48,0.00}{##1}}}
\expandafter\def\csname PY@tok@k\endcsname{\let\PY@bf=\textbf\def\PY@tc##1{\textcolor[rgb]{0.00,0.50,0.00}{##1}}}
\expandafter\def\csname PY@tok@kp\endcsname{\def\PY@tc##1{\textcolor[rgb]{0.00,0.50,0.00}{##1}}}
\expandafter\def\csname PY@tok@kt\endcsname{\def\PY@tc##1{\textcolor[rgb]{0.69,0.00,0.25}{##1}}}
\expandafter\def\csname PY@tok@o\endcsname{\def\PY@tc##1{\textcolor[rgb]{0.40,0.40,0.40}{##1}}}
\expandafter\def\csname PY@tok@ow\endcsname{\let\PY@bf=\textbf\def\PY@tc##1{\textcolor[rgb]{0.67,0.13,1.00}{##1}}}
\expandafter\def\csname PY@tok@nb\endcsname{\def\PY@tc##1{\textcolor[rgb]{0.00,0.50,0.00}{##1}}}
\expandafter\def\csname PY@tok@nf\endcsname{\def\PY@tc##1{\textcolor[rgb]{0.00,0.00,1.00}{##1}}}
\expandafter\def\csname PY@tok@nc\endcsname{\let\PY@bf=\textbf\def\PY@tc##1{\textcolor[rgb]{0.00,0.00,1.00}{##1}}}
\expandafter\def\csname PY@tok@nn\endcsname{\let\PY@bf=\textbf\def\PY@tc##1{\textcolor[rgb]{0.00,0.00,1.00}{##1}}}
\expandafter\def\csname PY@tok@ne\endcsname{\let\PY@bf=\textbf\def\PY@tc##1{\textcolor[rgb]{0.82,0.25,0.23}{##1}}}
\expandafter\def\csname PY@tok@nv\endcsname{\def\PY@tc##1{\textcolor[rgb]{0.10,0.09,0.49}{##1}}}
\expandafter\def\csname PY@tok@no\endcsname{\def\PY@tc##1{\textcolor[rgb]{0.53,0.00,0.00}{##1}}}
\expandafter\def\csname PY@tok@nl\endcsname{\def\PY@tc##1{\textcolor[rgb]{0.63,0.63,0.00}{##1}}}
\expandafter\def\csname PY@tok@ni\endcsname{\let\PY@bf=\textbf\def\PY@tc##1{\textcolor[rgb]{0.60,0.60,0.60}{##1}}}
\expandafter\def\csname PY@tok@na\endcsname{\def\PY@tc##1{\textcolor[rgb]{0.49,0.56,0.16}{##1}}}
\expandafter\def\csname PY@tok@nt\endcsname{\let\PY@bf=\textbf\def\PY@tc##1{\textcolor[rgb]{0.00,0.50,0.00}{##1}}}
\expandafter\def\csname PY@tok@nd\endcsname{\def\PY@tc##1{\textcolor[rgb]{0.67,0.13,1.00}{##1}}}
\expandafter\def\csname PY@tok@s\endcsname{\def\PY@tc##1{\textcolor[rgb]{0.73,0.13,0.13}{##1}}}
\expandafter\def\csname PY@tok@sd\endcsname{\let\PY@it=\textit\def\PY@tc##1{\textcolor[rgb]{0.73,0.13,0.13}{##1}}}
\expandafter\def\csname PY@tok@si\endcsname{\let\PY@bf=\textbf\def\PY@tc##1{\textcolor[rgb]{0.73,0.40,0.53}{##1}}}
\expandafter\def\csname PY@tok@se\endcsname{\let\PY@bf=\textbf\def\PY@tc##1{\textcolor[rgb]{0.73,0.40,0.13}{##1}}}
\expandafter\def\csname PY@tok@sr\endcsname{\def\PY@tc##1{\textcolor[rgb]{0.73,0.40,0.53}{##1}}}
\expandafter\def\csname PY@tok@ss\endcsname{\def\PY@tc##1{\textcolor[rgb]{0.10,0.09,0.49}{##1}}}
\expandafter\def\csname PY@tok@sx\endcsname{\def\PY@tc##1{\textcolor[rgb]{0.00,0.50,0.00}{##1}}}
\expandafter\def\csname PY@tok@m\endcsname{\def\PY@tc##1{\textcolor[rgb]{0.40,0.40,0.40}{##1}}}
\expandafter\def\csname PY@tok@gh\endcsname{\let\PY@bf=\textbf\def\PY@tc##1{\textcolor[rgb]{0.00,0.00,0.50}{##1}}}
\expandafter\def\csname PY@tok@gu\endcsname{\let\PY@bf=\textbf\def\PY@tc##1{\textcolor[rgb]{0.50,0.00,0.50}{##1}}}
\expandafter\def\csname PY@tok@gd\endcsname{\def\PY@tc##1{\textcolor[rgb]{0.63,0.00,0.00}{##1}}}
\expandafter\def\csname PY@tok@gi\endcsname{\def\PY@tc##1{\textcolor[rgb]{0.00,0.63,0.00}{##1}}}
\expandafter\def\csname PY@tok@gr\endcsname{\def\PY@tc##1{\textcolor[rgb]{1.00,0.00,0.00}{##1}}}
\expandafter\def\csname PY@tok@ge\endcsname{\let\PY@it=\textit}
\expandafter\def\csname PY@tok@gs\endcsname{\let\PY@bf=\textbf}
\expandafter\def\csname PY@tok@gp\endcsname{\let\PY@bf=\textbf\def\PY@tc##1{\textcolor[rgb]{0.00,0.00,0.50}{##1}}}
\expandafter\def\csname PY@tok@go\endcsname{\def\PY@tc##1{\textcolor[rgb]{0.53,0.53,0.53}{##1}}}
\expandafter\def\csname PY@tok@gt\endcsname{\def\PY@tc##1{\textcolor[rgb]{0.00,0.27,0.87}{##1}}}
\expandafter\def\csname PY@tok@err\endcsname{\def\PY@bc##1{\setlength{\fboxsep}{0pt}\fcolorbox[rgb]{1.00,0.00,0.00}{1,1,1}{\strut ##1}}}
\expandafter\def\csname PY@tok@kc\endcsname{\let\PY@bf=\textbf\def\PY@tc##1{\textcolor[rgb]{0.00,0.50,0.00}{##1}}}
\expandafter\def\csname PY@tok@kd\endcsname{\let\PY@bf=\textbf\def\PY@tc##1{\textcolor[rgb]{0.00,0.50,0.00}{##1}}}
\expandafter\def\csname PY@tok@kn\endcsname{\let\PY@bf=\textbf\def\PY@tc##1{\textcolor[rgb]{0.00,0.50,0.00}{##1}}}
\expandafter\def\csname PY@tok@kr\endcsname{\let\PY@bf=\textbf\def\PY@tc##1{\textcolor[rgb]{0.00,0.50,0.00}{##1}}}
\expandafter\def\csname PY@tok@bp\endcsname{\def\PY@tc##1{\textcolor[rgb]{0.00,0.50,0.00}{##1}}}
\expandafter\def\csname PY@tok@fm\endcsname{\def\PY@tc##1{\textcolor[rgb]{0.00,0.00,1.00}{##1}}}
\expandafter\def\csname PY@tok@vc\endcsname{\def\PY@tc##1{\textcolor[rgb]{0.10,0.09,0.49}{##1}}}
\expandafter\def\csname PY@tok@vg\endcsname{\def\PY@tc##1{\textcolor[rgb]{0.10,0.09,0.49}{##1}}}
\expandafter\def\csname PY@tok@vi\endcsname{\def\PY@tc##1{\textcolor[rgb]{0.10,0.09,0.49}{##1}}}
\expandafter\def\csname PY@tok@vm\endcsname{\def\PY@tc##1{\textcolor[rgb]{0.10,0.09,0.49}{##1}}}
\expandafter\def\csname PY@tok@sa\endcsname{\def\PY@tc##1{\textcolor[rgb]{0.73,0.13,0.13}{##1}}}
\expandafter\def\csname PY@tok@sb\endcsname{\def\PY@tc##1{\textcolor[rgb]{0.73,0.13,0.13}{##1}}}
\expandafter\def\csname PY@tok@sc\endcsname{\def\PY@tc##1{\textcolor[rgb]{0.73,0.13,0.13}{##1}}}
\expandafter\def\csname PY@tok@dl\endcsname{\def\PY@tc##1{\textcolor[rgb]{0.73,0.13,0.13}{##1}}}
\expandafter\def\csname PY@tok@s2\endcsname{\def\PY@tc##1{\textcolor[rgb]{0.73,0.13,0.13}{##1}}}
\expandafter\def\csname PY@tok@sh\endcsname{\def\PY@tc##1{\textcolor[rgb]{0.73,0.13,0.13}{##1}}}
\expandafter\def\csname PY@tok@s1\endcsname{\def\PY@tc##1{\textcolor[rgb]{0.73,0.13,0.13}{##1}}}
\expandafter\def\csname PY@tok@mb\endcsname{\def\PY@tc##1{\textcolor[rgb]{0.40,0.40,0.40}{##1}}}
\expandafter\def\csname PY@tok@mf\endcsname{\def\PY@tc##1{\textcolor[rgb]{0.40,0.40,0.40}{##1}}}
\expandafter\def\csname PY@tok@mh\endcsname{\def\PY@tc##1{\textcolor[rgb]{0.40,0.40,0.40}{##1}}}
\expandafter\def\csname PY@tok@mi\endcsname{\def\PY@tc##1{\textcolor[rgb]{0.40,0.40,0.40}{##1}}}
\expandafter\def\csname PY@tok@il\endcsname{\def\PY@tc##1{\textcolor[rgb]{0.40,0.40,0.40}{##1}}}
\expandafter\def\csname PY@tok@mo\endcsname{\def\PY@tc##1{\textcolor[rgb]{0.40,0.40,0.40}{##1}}}
\expandafter\def\csname PY@tok@ch\endcsname{\let\PY@it=\textit\def\PY@tc##1{\textcolor[rgb]{0.25,0.50,0.50}{##1}}}
\expandafter\def\csname PY@tok@cm\endcsname{\let\PY@it=\textit\def\PY@tc##1{\textcolor[rgb]{0.25,0.50,0.50}{##1}}}
\expandafter\def\csname PY@tok@cpf\endcsname{\let\PY@it=\textit\def\PY@tc##1{\textcolor[rgb]{0.25,0.50,0.50}{##1}}}
\expandafter\def\csname PY@tok@c1\endcsname{\let\PY@it=\textit\def\PY@tc##1{\textcolor[rgb]{0.25,0.50,0.50}{##1}}}
\expandafter\def\csname PY@tok@cs\endcsname{\let\PY@it=\textit\def\PY@tc##1{\textcolor[rgb]{0.25,0.50,0.50}{##1}}}

\def\PYZbs{\char`\\}
\def\PYZus{\char`\_}
\def\PYZob{\char`\{}
\def\PYZcb{\char`\}}
\def\PYZca{\char`\^}
\def\PYZam{\char`\&}
\def\PYZlt{\char`\<}
\def\PYZgt{\char`\>}
\def\PYZsh{\char`\#}
\def\PYZpc{\char`\%}
\def\PYZdl{\char`\$}
\def\PYZhy{\char`\-}
\def\PYZsq{\char`\'}
\def\PYZdq{\char`\"}
\def\PYZti{\char`\~}
% for compatibility with earlier versions
\def\PYZat{@}
\def\PYZlb{[}
\def\PYZrb{]}
\makeatother


    % Exact colors from NB
    \definecolor{incolor}{rgb}{0.0, 0.0, 0.5}
    \definecolor{outcolor}{rgb}{0.545, 0.0, 0.0}



    
    % Prevent overflowing lines due to hard-to-break entities
    \sloppy 
    % Setup hyperref package
    \hypersetup{
      breaklinks=true,  % so long urls are correctly broken across lines
      colorlinks=true,
      urlcolor=urlcolor,
      linkcolor=linkcolor,
      citecolor=citecolor,
      }
    % Slightly bigger margins than the latex defaults
    
    \geometry{verbose,tmargin=1in,bmargin=1in,lmargin=1in,rmargin=1in, left=1cm, right=1cm}
    
    

    \begin{document}
    
    
    \maketitle
    
    

    
    \subsubsection*{Import packages}\label{import-packages}

    \begin{Verbatim}[commandchars=\\\{\}]
{\color{incolor}In [{\color{incolor}23}]:} \PY{k+kn}{import} \PY{n+nn}{datascience}
         \PY{k+kn}{from} \PY{n+nn}{datascience} \PY{k}{import} \PY{o}{*}
         \PY{k+kn}{import} \PY{n+nn}{numpy} \PY{k}{as} \PY{n+nn}{np}
         \PY{k+kn}{import} \PY{n+nn}{pandas} \PY{k}{as} \PY{n+nn}{pd}
         \PY{k+kn}{import} \PY{n+nn}{graphviz}
         \PY{k+kn}{from} \PY{n+nn}{sklearn}\PY{n+nn}{.}\PY{n+nn}{model\PYZus{}selection} \PY{k}{import} \PY{n}{train\PYZus{}test\PYZus{}split}
         \PY{k+kn}{from} \PY{n+nn}{sklearn}\PY{n+nn}{.}\PY{n+nn}{pipeline} \PY{k}{import} \PY{n}{Pipeline}
         \PY{k+kn}{from} \PY{n+nn}{sklearn} \PY{k}{import} \PY{n}{tree}
         \PY{k+kn}{from} \PY{n+nn}{sklearn}\PY{n+nn}{.}\PY{n+nn}{tree} \PY{k}{import} \PY{n}{DecisionTreeClassifier}
         \PY{k+kn}{from} \PY{n+nn}{sklearn} \PY{k}{import} \PY{n}{metrics}
         
\end{Verbatim}

    \subsubsection*{Read in the dataset with
pandas}\label{read-in-the-dataset-with-pandas}


    \begin{Verbatim}[commandchars=\\\{\}]
{\color{incolor}In [{\color{incolor}63}]:} \PY{n}{Ben\PYZus{}Pass}\PY{o}{=} \PY{n}{pd}\PY{o}{.}\PY{n}{read\PYZus{}csv}\PY{p}{(}\PY{l+s+s1}{\PYZsq{}}\PY{l+s+s1}{Ben\PYZhy{}NE\PYZhy{}9\PYZhy{}10\PYZhy{}2015\PYZhy{}pass\PYZhy{}6.csv}\PY{l+s+s1}{\PYZsq{}}\PY{p}{,} \PY{n}{sep}\PY{o}{=}\PY{l+s+s2}{\PYZdq{}}\PY{l+s+s2}{,}\PY{l+s+s2}{\PYZdq{}}\PY{p}{)}
         \PY{n}{Ben\PYZus{}Pass\PYZus{}2} \PY{o}{=} \PY{n}{pd}\PY{o}{.}\PY{n}{read\PYZus{}csv}\PY{p}{(}\PY{l+s+s1}{\PYZsq{}}\PY{l+s+s1}{Ben\PYZhy{}NE\PYZhy{}9\PYZhy{}10\PYZhy{}2015\PYZhy{}pass\PYZhy{}4\PYZhy{}1.csv}\PY{l+s+s1}{\PYZsq{}}\PY{p}{,} \PY{n}{sep}\PY{o}{=}\PY{l+s+s1}{\PYZsq{}}\PY{l+s+s1}{,}\PY{l+s+s1}{\PYZsq{}}\PY{p}{)}
\end{Verbatim}

    \begin{Verbatim}[commandchars=\\\{\}]
{\color{incolor}In [{\color{incolor}64}]:} \PY{n}{Ben\PYZus{}Pass}\PY{o}{.}\PY{n}{head}\PY{p}{(}\PY{p}{)}
\end{Verbatim}

\begin{tabular}{lrrrrrr}
	\toprule
	{} &  down &  ydstogo &  Yards.Gained.PrevPlay &  AirYards &  PassLocation &  PassOutcome \\
	\midrule
	0 &     1 &       10 &                     18 &        -4 &             1 &            1 \\
	1 &     1 &       10 &                      0 &         9 &             1 &            1 \\
	2 &     3 &       22 &                      6 &         1 &             1 &            1 \\
	3 &     1 &       10 &                      0 &         7 &            -1 &            1 \\
	4 &     1 &       10 &                     13 &         6 &            -1 &            1 \\
	\bottomrule
\end{tabular}
            
    \begin{Verbatim}[commandchars=\\\{\}]
{\color{incolor}In [{\color{incolor}25}]:} \PY{n}{Ben\PYZus{}Pass\PYZus{}2}\PY{o}{.}\PY{n}{head}\PY{p}{(}\PY{p}{)}
\end{Verbatim}

\begin{tabular}{lrrrrrr}
	\toprule
	{} &  down &  ydstogo &  Yards.Gained.PrevPlay &  AirYards &  PassLocation &  PassOutcome \\
	\midrule
	0 &     1 &       10 &                     18 &        -4 &             1 &            1 \\
	1 &     1 &       10 &                      0 &         9 &             1 &            1 \\
	2 &     3 &       22 &                      6 &         1 &             1 &            1 \\
	3 &     1 &       10 &                      0 &         7 &            -1 &            1 \\
	4 &     1 &       10 &                     13 &         6 &            -1 &            1 \\
	\bottomrule
\end{tabular}
            
    \subsubsection*{Conduct Correlation
Analysis}\label{conduct-correlation-analysis}

    \begin{Verbatim}[commandchars=\\\{\}]
{\color{incolor}In [{\color{incolor}65}]:} \PY{n}{Ben\PYZus{}Pass}\PY{o}{.}\PY{n}{corr}\PY{p}{(}\PY{n}{method}\PY{o}{=}\PY{l+s+s1}{\PYZsq{}}\PY{l+s+s1}{pearson}\PY{l+s+s1}{\PYZsq{}}\PY{p}{)}
\end{Verbatim}

\begin{tabular}{lrrrrrr}
	\toprule
	{} &      down &   ydstogo &  Yards.PrevPlay &  AirYards &  PassLocation &  PassOutcome \\
	\midrule
	down                  &  1.000000 & -0.293906 &              -0.306207 & -0.032894 &     -0.060043 &    -0.076448 \\
	ydstogo               & -0.293906 &  1.000000 &              -0.060572 &  0.091813 &     -0.058176 &     0.249902 \\
	Yards.PrevPlay & -0.306207 & -0.060572 &               1.000000 &  0.022748 &     -0.081227 &     0.144037 \\
	AirYards              & -0.032894 &  0.091813 &               0.022748 &  1.000000 &      0.056054 &    -0.286445 \\
	PassLocation          & -0.060043 & -0.058176 &              -0.081227 &  0.056054 &      1.000000 &    -0.223061 \\
	PassOutcome           & -0.076448 &  0.249902 &               0.144037 & -0.286445 &     -0.223061 &     1.000000 \\
	\bottomrule
\end{tabular}

            
    \begin{Verbatim}[commandchars=\\\{\}]
{\color{incolor}In [{\color{incolor}27}]:} \PY{n}{Ben\PYZus{}Pass\PYZus{}2}\PY{o}{.}\PY{n}{corr}\PY{p}{(}\PY{n}{method}\PY{o}{=}\PY{l+s+s1}{\PYZsq{}}\PY{l+s+s1}{pearson}\PY{l+s+s1}{\PYZsq{}}\PY{p}{)}
\end{Verbatim}

\begin{tabular}{lrrrrrrr}
	\toprule
	{} &      down &   ydstogo &  Yards.PrevPlay &  Yards.Gained &  AirYards &  PassLocation &  PassOutcome \\
	\midrule
	down                  &  1.000000 & -0.293906 &              -0.306207 &     -0.054182 & -0.032894 &     -0.060043 &    -0.076448 \\
	ydstogo               & -0.293906 &  1.000000 &              -0.060572 &      0.222043 &  0.091813 &     -0.058176 &     0.249902 \\
	Yards.PrevPlay & -0.306207 & -0.060572 &               1.000000 &      0.131102 &  0.022748 &     -0.081227 &     0.144037 \\
	Yards.Gained          & -0.054182 &  0.222043 &               0.131102 &      1.000000 &  0.363336 &      0.019662 &     0.537075 \\
	AirYards              & -0.032894 &  0.091813 &               0.022748 &      0.363336 &  1.000000 &      0.056054 &    -0.286445 \\
	PassLocation          & -0.060043 & -0.058176 &              -0.081227 &      0.019662 &  0.056054 &      1.000000 &    -0.223061 \\
	PassOutcome           & -0.076448 &  0.249902 &               0.144037 &      0.537075 & -0.286445 &     -0.223061 &     1.000000 \\
	\bottomrule
\end{tabular}

            
    \subsubsection*{Build Decision Trees}\label{build-decision-trees}

    \begin{Verbatim}[commandchars=\\\{\}]
{\color{incolor}In [{\color{incolor}113}]:} \PY{k}{def} \PY{n+nf}{evaluate}\PY{p}{(}\PY{n}{dataset}\PY{p}{,} \PY{n}{seed}\PY{p}{,} \PY{n}{arg}\PY{p}{)}\PY{p}{:}
              \PY{n}{X} \PY{o}{=} \PY{n}{dataset}\PY{o}{.}\PY{n}{values}\PY{p}{[}\PY{p}{:}\PY{p}{,} \PY{p}{:}\PY{o}{\PYZhy{}}\PY{l+m+mi}{1}\PY{p}{]}
              \PY{n}{y} \PY{o}{=} \PY{n}{dataset}\PY{o}{.}\PY{n}{values}\PY{p}{[}\PY{p}{:}\PY{p}{,} \PY{o}{\PYZhy{}}\PY{l+m+mi}{1}\PY{p}{]}
              \PY{n}{X\PYZus{}train}\PY{p}{,} \PY{n}{X\PYZus{}test}\PY{p}{,} \PY{n}{y\PYZus{}train}\PY{p}{,} \PY{n}{y\PYZus{}test} \PY{o}{=} \PY{n}{train\PYZus{}test\PYZus{}split}\PY{p}{(}\PY{n}{X}\PY{p}{,} \PY{n}{y}\PY{p}{,} \PY{n}{test\PYZus{}size} \PY{o}{=} \PY{l+m+mf}{0.2}\PY{p}{,} \PY{n}{random\PYZus{}state} \PY{o}{=} \PY{n}{seed}\PY{p}{)}
              \PY{n}{clf} \PY{o}{=} \PY{n}{DecisionTreeClassifier}\PY{p}{(}\PY{n}{criterion} \PY{o}{=} \PY{l+s+s1}{\PYZsq{}}\PY{l+s+s1}{entropy}\PY{l+s+s1}{\PYZsq{}}\PY{p}{,} \PY{n}{random\PYZus{}state} \PY{o}{=} \PY{l+m+mi}{100}\PY{p}{,}
                                         \PY{n}{max\PYZus{}depth}\PY{o}{=}\PY{l+m+mi}{5}\PY{p}{,} \PY{n}{min\PYZus{}samples\PYZus{}leaf}\PY{o}{=}\PY{l+m+mi}{3}\PY{p}{)}
              \PY{n}{clf}\PY{o}{.}\PY{n}{fit}\PY{p}{(}\PY{n}{X\PYZus{}train}\PY{p}{,} \PY{n}{y\PYZus{}train}\PY{p}{)}
              \PY{n}{dot\PYZus{}data}\PY{o}{=} \PY{n}{tree}\PY{o}{.}\PY{n}{export\PYZus{}graphviz}\PY{p}{(}\PY{n}{clf}\PY{p}{,} \PY{n}{out\PYZus{}file}\PY{o}{=}\PY{k+kc}{None}\PY{p}{,} \PY{n}{feature\PYZus{}names}\PY{o}{=}\PY{n}{dataset}\PY{o}{.}\PY{n}{columns}\PY{p}{[}\PY{p}{:}\PY{o}{\PYZhy{}}\PY{l+m+mi}{1}\PY{p}{]}\PY{p}{)}
              \PY{n}{graph} \PY{o}{=} \PY{n}{graphviz}\PY{o}{.}\PY{n}{Source}\PY{p}{(}\PY{n}{dot\PYZus{}data}\PY{p}{)}
              \PY{n}{graph}\PY{o}{.}\PY{n}{render}\PY{p}{(}\PY{l+s+s1}{\PYZsq{}}\PY{l+s+s1}{BenPass}\PY{l+s+si}{\PYZob{}\PYZcb{}}\PY{l+s+s1}{\PYZsq{}}\PY{o}{.}\PY{n}{format}\PY{p}{(}\PY{n+nb}{str}\PY{p}{(}\PY{n}{arg}\PY{p}{)}\PY{p}{)}\PY{p}{)}
              \PY{n}{predicted\PYZus{}completion} \PY{o}{=} \PY{n}{clf}\PY{o}{.}\PY{n}{predict}\PY{p}{(}\PY{n}{X\PYZus{}test}\PY{p}{)}
              \PY{n+nb}{print}\PY{p}{(}\PY{l+s+s1}{\PYZsq{}}\PY{l+s+s1}{Ben Pass dataset }\PY{l+s+si}{\PYZob{}\PYZcb{}}\PY{l+s+s1}{ classification report: }\PY{l+s+se}{\PYZbs{}n}\PY{l+s+s1}{\PYZsq{}}\PY{o}{.}\PY{n}{format}\PY{p}{(}\PY{n+nb}{str}\PY{p}{(}\PY{n}{arg}\PY{p}{)}\PY{p}{)}\PY{p}{,}
               \PY{n}{metrics}\PY{o}{.}\PY{n}{classification\PYZus{}report}\PY{p}{(}\PY{n}{y\PYZus{}true}\PY{o}{=}\PY{n}{y\PYZus{}test}\PY{p}{,} \PY{n}{y\PYZus{}pred}\PY{o}{=}\PY{n}{predicted\PYZus{}completion}\PY{p}{)}\PY{p}{)}
\end{Verbatim}

    \begin{Verbatim}[commandchars=\\\{\}]
{\color{incolor}In [{\color{incolor}114}]:} \PY{n}{evaluate}\PY{p}{(}\PY{n}{Ben\PYZus{}Pass}\PY{p}{,} \PY{l+m+mi}{2018}\PY{p}{,} \PY{l+m+mi}{1}\PY{p}{)}
          \PY{n}{evaluate}\PY{p}{(}\PY{n}{Ben\PYZus{}Pass\PYZus{}2}\PY{p}{,} \PY{l+m+mi}{2018}\PY{p}{,} \PY{l+m+mi}{2}\PY{p}{)}
\end{Verbatim}

    \begin{Verbatim}[commandchars=\\\{\}]
Ben Pass dataset 1 classification report: 
               precision    recall  f1-score   support

           0       0.75      0.75      0.75         4
           1       0.90      0.90      0.90        10

   micro avg       0.86      0.86      0.86        14
   macro avg       0.82      0.82      0.82        14
weighted avg       0.86      0.86      0.86        14

Ben Pass dataset 2 classification report: 
               precision    recall  f1-score   support

           0       1.00      1.00      1.00         4
           1       1.00      1.00      1.00        10

   micro avg       1.00      1.00      1.00        14
   macro avg       1.00      1.00      1.00        14
weighted avg       1.00      1.00      1.00        14


    \end{Verbatim}

    So from the classification report we could see that the Decision Tree
built for the second dataset (\textbf{with Yard.Gained}) has no
misclassified instances, reaching 100\% accuracy score.

  
  
            
    \subsubsection*{Discussion of the Correlation and DecisionTree}\label{discussion}

    \subsubsection*{\texorpdfstring{Ben\_Pass\_1 (Without the
\texttt{Yard.Gained}
variable)}{Ben\_Pass\_1 (Without the Yard.Gained variable)}}\label{ben_pass_1-without-the-yard.gained-variable}

\begin{figure}[H]
\centering
\includegraphics{2.png}
\caption{}
\end{figure}

\subsubsection*{\texorpdfstring{Ben\_Pass\_2 (With the \texttt{Yard.Gained}
variable)}{Ben\_Pass\_2 (With the Yard.Gained variable)}}\label{ben_pass_2-with-the-yard.gained-variable}

\begin{figure}[H]
\centering
\includegraphics{1.png}
\caption{}
\end{figure}

    The second Ben Pass dataset that includes the \texttt{Yards.Gained}
variable stop after two splits (if the random state is changed to
another number, then the tree actually stops after one single split),
the reason for which is:\\

\noindent Revisit the correlation matrix:

    \begin{Verbatim}[commandchars=\\\{\}]
{\color{incolor}In [{\color{incolor}87}]:} \PY{n}{Ben\PYZus{}Pass\PYZus{}2}\PY{o}{.}\PY{n}{corr}\PY{p}{(}\PY{n}{method}\PY{o}{=}\PY{l+s+s1}{\PYZsq{}}\PY{l+s+s1}{pearson}\PY{l+s+s1}{\PYZsq{}}\PY{p}{)}
\end{Verbatim}

\begin{tabular}{lrrrrrrr}
	\toprule
	{} &      down &   ydstogo &  Yards.PrevPlay &  Yards.Gained &  AirYards &  PassLocation &  PassOutcome \\
	\midrule
	down                  &  1.000000 & -0.293906 &              -0.306207 &     -0.054182 & -0.032894 &     -0.060043 &    -0.076448 \\
	ydstogo               & -0.293906 &  1.000000 &              -0.060572 &      0.222043 &  0.091813 &     -0.058176 &     0.249902 \\
	Yards.PrevPlay & -0.306207 & -0.060572 &               1.000000 &      0.131102 &  0.022748 &     -0.081227 &     0.144037 \\
	Yards.Gained          & -0.054182 &  0.222043 &               0.131102 &      1.000000 &  0.363336 &      0.019662 &     0.537075 \\
	AirYards              & -0.032894 &  0.091813 &               0.022748 &      0.363336 &  1.000000 &      0.056054 &    -0.286445 \\
	PassLocation          & -0.060043 & -0.058176 &              -0.081227 &      0.019662 &  0.056054 &      1.000000 &    -0.223061 \\
	PassOutcome           & -0.076448 &  0.249902 &               0.144037 &      0.537075 & -0.286445 &     -0.223061 &     1.000000 \\
	\bottomrule
\end{tabular}\mbox{}\mbox{}


\subsubsection*{Conclusion:}

            
\noindent    from the correlation matrix above, we could observe that the correlation
coefficient between \emph{Yards.Gained} and \emph{PassOutcome} is
0.537075 (positively correlated), thus we should be able to conclude
that \emph{Yards.Gained} has a significant influence on the
\emph{PassOutcome}. The more correlated a variable is to the target
variable, the higher up it should be in the constructed Decision Tree,
hence explains that the \emph{Yards.Gained} is the root node, and the
second largest correlation coefficient among the other variables to
\emph{PassOutcome} is \emph{AirYards} as 0.286445 (negatively
correlated), so the first child right after the root node is
\emph{AirYards}.


    % Add a bibliography block to the postdoc
    
    
    
    \end{document}
