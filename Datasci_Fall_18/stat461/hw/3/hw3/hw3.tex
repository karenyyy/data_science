
% Default to the notebook output style

    


% Inherit from the specified cell style.




    
\documentclass[11pt]{article}

    
    
    \usepackage[T1]{fontenc}
    % Nicer default font (+ math font) than Computer Modern for most use cases
    \usepackage{mathpazo}

    % Basic figure setup, for now with no caption control since it's done
    % automatically by Pandoc (which extracts ![](path) syntax from Markdown).
    \usepackage{graphicx}
    % We will generate all images so they have a width \maxwidth. This means
    % that they will get their normal width if they fit onto the page, but
    % are scaled down if they would overflow the margins.
    \makeatletter
    \def\maxwidth{\ifdim\Gin@nat@width>\linewidth\linewidth
    \else\Gin@nat@width\fi}
    \makeatother
    \let\Oldincludegraphics\includegraphics
    % Set max figure width to be 80% of text width, for now hardcoded.
    \renewcommand{\includegraphics}[1]{\Oldincludegraphics[width=.8\maxwidth]{#1}}
    % Ensure that by default, figures have no caption (until we provide a
    % proper Figure object with a Caption API and a way to capture that
    % in the conversion process - todo).
    \usepackage{caption}
    \DeclareCaptionLabelFormat{nolabel}{}
    \captionsetup{labelformat=nolabel}

    \usepackage{adjustbox} % Used to constrain images to a maximum size 
    \usepackage{xcolor} % Allow colors to be defined
    \usepackage{enumerate} % Needed for markdown enumerations to work
    \usepackage{geometry} % Used to adjust the document margins
    \usepackage{amsmath} % Equations
    \usepackage{amssymb} % Equations
    \usepackage{textcomp} % defines textquotesingle
    % Hack from http://tex.stackexchange.com/a/47451/13684:
    \AtBeginDocument{%
        \def\PYZsq{\textquotesingle}% Upright quotes in Pygmentized code
    }
    \usepackage{upquote} % Upright quotes for verbatim code
    \usepackage{eurosym} % defines \euro
    \usepackage[mathletters]{ucs} % Extended unicode (utf-8) support
    \usepackage[utf8x]{inputenc} % Allow utf-8 characters in the tex document
    \usepackage{fancyvrb} % verbatim replacement that allows latex
    \usepackage{grffile} % extends the file name processing of package graphics 
                         % to support a larger range 
    % The hyperref package gives us a pdf with properly built
    % internal navigation ('pdf bookmarks' for the table of contents,
    % internal cross-reference links, web links for URLs, etc.)
    \usepackage{hyperref}
    \usepackage{longtable} % longtable support required by pandoc >1.10
    \usepackage{booktabs}  % table support for pandoc > 1.12.2
    \usepackage[inline]{enumitem} % IRkernel/repr support (it uses the enumerate* environment)
    \usepackage[normalem]{ulem} % ulem is needed to support strikethroughs (\sout)
                                % normalem makes italics be italics, not underlines
    \usepackage{mathrsfs}
    

    
    
    % Colors for the hyperref package
    \definecolor{urlcolor}{rgb}{0,.145,.698}
    \definecolor{linkcolor}{rgb}{.71,0.21,0.01}
    \definecolor{citecolor}{rgb}{.12,.54,.11}

    % ANSI colors
    \definecolor{ansi-black}{HTML}{3E424D}
    \definecolor{ansi-black-intense}{HTML}{282C36}
    \definecolor{ansi-red}{HTML}{E75C58}
    \definecolor{ansi-red-intense}{HTML}{B22B31}
    \definecolor{ansi-green}{HTML}{00A250}
    \definecolor{ansi-green-intense}{HTML}{007427}
    \definecolor{ansi-yellow}{HTML}{DDB62B}
    \definecolor{ansi-yellow-intense}{HTML}{B27D12}
    \definecolor{ansi-blue}{HTML}{208FFB}
    \definecolor{ansi-blue-intense}{HTML}{0065CA}
    \definecolor{ansi-magenta}{HTML}{D160C4}
    \definecolor{ansi-magenta-intense}{HTML}{A03196}
    \definecolor{ansi-cyan}{HTML}{60C6C8}
    \definecolor{ansi-cyan-intense}{HTML}{258F8F}
    \definecolor{ansi-white}{HTML}{C5C1B4}
    \definecolor{ansi-white-intense}{HTML}{A1A6B2}
    \definecolor{ansi-default-inverse-fg}{HTML}{FFFFFF}
    \definecolor{ansi-default-inverse-bg}{HTML}{000000}

    % commands and environments needed by pandoc snippets
    % extracted from the output of `pandoc -s`
    \providecommand{\tightlist}{%
      \setlength{\itemsep}{0pt}\setlength{\parskip}{0pt}}
    \DefineVerbatimEnvironment{Highlighting}{Verbatim}{commandchars=\\\{\}}
    % Add ',fontsize=\small' for more characters per line
    \newenvironment{Shaded}{}{}
    \newcommand{\KeywordTok}[1]{\textcolor[rgb]{0.00,0.44,0.13}{\textbf{{#1}}}}
    \newcommand{\DataTypeTok}[1]{\textcolor[rgb]{0.56,0.13,0.00}{{#1}}}
    \newcommand{\DecValTok}[1]{\textcolor[rgb]{0.25,0.63,0.44}{{#1}}}
    \newcommand{\BaseNTok}[1]{\textcolor[rgb]{0.25,0.63,0.44}{{#1}}}
    \newcommand{\FloatTok}[1]{\textcolor[rgb]{0.25,0.63,0.44}{{#1}}}
    \newcommand{\CharTok}[1]{\textcolor[rgb]{0.25,0.44,0.63}{{#1}}}
    \newcommand{\StringTok}[1]{\textcolor[rgb]{0.25,0.44,0.63}{{#1}}}
    \newcommand{\CommentTok}[1]{\textcolor[rgb]{0.38,0.63,0.69}{\textit{{#1}}}}
    \newcommand{\OtherTok}[1]{\textcolor[rgb]{0.00,0.44,0.13}{{#1}}}
    \newcommand{\AlertTok}[1]{\textcolor[rgb]{1.00,0.00,0.00}{\textbf{{#1}}}}
    \newcommand{\FunctionTok}[1]{\textcolor[rgb]{0.02,0.16,0.49}{{#1}}}
    \newcommand{\RegionMarkerTok}[1]{{#1}}
    \newcommand{\ErrorTok}[1]{\textcolor[rgb]{1.00,0.00,0.00}{\textbf{{#1}}}}
    \newcommand{\NormalTok}[1]{{#1}}
    
    % Additional commands for more recent versions of Pandoc
    \newcommand{\ConstantTok}[1]{\textcolor[rgb]{0.53,0.00,0.00}{{#1}}}
    \newcommand{\SpecialCharTok}[1]{\textcolor[rgb]{0.25,0.44,0.63}{{#1}}}
    \newcommand{\VerbatimStringTok}[1]{\textcolor[rgb]{0.25,0.44,0.63}{{#1}}}
    \newcommand{\SpecialStringTok}[1]{\textcolor[rgb]{0.73,0.40,0.53}{{#1}}}
    \newcommand{\ImportTok}[1]{{#1}}
    \newcommand{\DocumentationTok}[1]{\textcolor[rgb]{0.73,0.13,0.13}{\textit{{#1}}}}
    \newcommand{\AnnotationTok}[1]{\textcolor[rgb]{0.38,0.63,0.69}{\textbf{\textit{{#1}}}}}
    \newcommand{\CommentVarTok}[1]{\textcolor[rgb]{0.38,0.63,0.69}{\textbf{\textit{{#1}}}}}
    \newcommand{\VariableTok}[1]{\textcolor[rgb]{0.10,0.09,0.49}{{#1}}}
    \newcommand{\ControlFlowTok}[1]{\textcolor[rgb]{0.00,0.44,0.13}{\textbf{{#1}}}}
    \newcommand{\OperatorTok}[1]{\textcolor[rgb]{0.40,0.40,0.40}{{#1}}}
    \newcommand{\BuiltInTok}[1]{{#1}}
    \newcommand{\ExtensionTok}[1]{{#1}}
    \newcommand{\PreprocessorTok}[1]{\textcolor[rgb]{0.74,0.48,0.00}{{#1}}}
    \newcommand{\AttributeTok}[1]{\textcolor[rgb]{0.49,0.56,0.16}{{#1}}}
    \newcommand{\InformationTok}[1]{\textcolor[rgb]{0.38,0.63,0.69}{\textbf{\textit{{#1}}}}}
    \newcommand{\WarningTok}[1]{\textcolor[rgb]{0.38,0.63,0.69}{\textbf{\textit{{#1}}}}}
    
    
    % Define a nice break command that doesn't care if a line doesn't already
    % exist.
    \def\br{\hspace*{\fill} \\* }
    % Math Jax compatibility definitions
    \def\gt{>}
    \def\lt{<}
    \let\Oldtex\TeX
    \let\Oldlatex\LaTeX
    \renewcommand{\TeX}{\textrm{\Oldtex}}
    \renewcommand{\LaTeX}{\textrm{\Oldlatex}}
    % Document parameters
    % Document title
    \title{Assignment 3}
    
    \author{Jiarong Ye}
    
    
    

    % Pygments definitions
    
\makeatletter
\def\PY@reset{\let\PY@it=\relax \let\PY@bf=\relax%
    \let\PY@ul=\relax \let\PY@tc=\relax%
    \let\PY@bc=\relax \let\PY@ff=\relax}
\def\PY@tok#1{\csname PY@tok@#1\endcsname}
\def\PY@toks#1+{\ifx\relax#1\empty\else%
    \PY@tok{#1}\expandafter\PY@toks\fi}
\def\PY@do#1{\PY@bc{\PY@tc{\PY@ul{%
    \PY@it{\PY@bf{\PY@ff{#1}}}}}}}
\def\PY#1#2{\PY@reset\PY@toks#1+\relax+\PY@do{#2}}

\expandafter\def\csname PY@tok@w\endcsname{\def\PY@tc##1{\textcolor[rgb]{0.73,0.73,0.73}{##1}}}
\expandafter\def\csname PY@tok@c\endcsname{\let\PY@it=\textit\def\PY@tc##1{\textcolor[rgb]{0.25,0.50,0.50}{##1}}}
\expandafter\def\csname PY@tok@cp\endcsname{\def\PY@tc##1{\textcolor[rgb]{0.74,0.48,0.00}{##1}}}
\expandafter\def\csname PY@tok@k\endcsname{\let\PY@bf=\textbf\def\PY@tc##1{\textcolor[rgb]{0.00,0.50,0.00}{##1}}}
\expandafter\def\csname PY@tok@kp\endcsname{\def\PY@tc##1{\textcolor[rgb]{0.00,0.50,0.00}{##1}}}
\expandafter\def\csname PY@tok@kt\endcsname{\def\PY@tc##1{\textcolor[rgb]{0.69,0.00,0.25}{##1}}}
\expandafter\def\csname PY@tok@o\endcsname{\def\PY@tc##1{\textcolor[rgb]{0.40,0.40,0.40}{##1}}}
\expandafter\def\csname PY@tok@ow\endcsname{\let\PY@bf=\textbf\def\PY@tc##1{\textcolor[rgb]{0.67,0.13,1.00}{##1}}}
\expandafter\def\csname PY@tok@nb\endcsname{\def\PY@tc##1{\textcolor[rgb]{0.00,0.50,0.00}{##1}}}
\expandafter\def\csname PY@tok@nf\endcsname{\def\PY@tc##1{\textcolor[rgb]{0.00,0.00,1.00}{##1}}}
\expandafter\def\csname PY@tok@nc\endcsname{\let\PY@bf=\textbf\def\PY@tc##1{\textcolor[rgb]{0.00,0.00,1.00}{##1}}}
\expandafter\def\csname PY@tok@nn\endcsname{\let\PY@bf=\textbf\def\PY@tc##1{\textcolor[rgb]{0.00,0.00,1.00}{##1}}}
\expandafter\def\csname PY@tok@ne\endcsname{\let\PY@bf=\textbf\def\PY@tc##1{\textcolor[rgb]{0.82,0.25,0.23}{##1}}}
\expandafter\def\csname PY@tok@nv\endcsname{\def\PY@tc##1{\textcolor[rgb]{0.10,0.09,0.49}{##1}}}
\expandafter\def\csname PY@tok@no\endcsname{\def\PY@tc##1{\textcolor[rgb]{0.53,0.00,0.00}{##1}}}
\expandafter\def\csname PY@tok@nl\endcsname{\def\PY@tc##1{\textcolor[rgb]{0.63,0.63,0.00}{##1}}}
\expandafter\def\csname PY@tok@ni\endcsname{\let\PY@bf=\textbf\def\PY@tc##1{\textcolor[rgb]{0.60,0.60,0.60}{##1}}}
\expandafter\def\csname PY@tok@na\endcsname{\def\PY@tc##1{\textcolor[rgb]{0.49,0.56,0.16}{##1}}}
\expandafter\def\csname PY@tok@nt\endcsname{\let\PY@bf=\textbf\def\PY@tc##1{\textcolor[rgb]{0.00,0.50,0.00}{##1}}}
\expandafter\def\csname PY@tok@nd\endcsname{\def\PY@tc##1{\textcolor[rgb]{0.67,0.13,1.00}{##1}}}
\expandafter\def\csname PY@tok@s\endcsname{\def\PY@tc##1{\textcolor[rgb]{0.73,0.13,0.13}{##1}}}
\expandafter\def\csname PY@tok@sd\endcsname{\let\PY@it=\textit\def\PY@tc##1{\textcolor[rgb]{0.73,0.13,0.13}{##1}}}
\expandafter\def\csname PY@tok@si\endcsname{\let\PY@bf=\textbf\def\PY@tc##1{\textcolor[rgb]{0.73,0.40,0.53}{##1}}}
\expandafter\def\csname PY@tok@se\endcsname{\let\PY@bf=\textbf\def\PY@tc##1{\textcolor[rgb]{0.73,0.40,0.13}{##1}}}
\expandafter\def\csname PY@tok@sr\endcsname{\def\PY@tc##1{\textcolor[rgb]{0.73,0.40,0.53}{##1}}}
\expandafter\def\csname PY@tok@ss\endcsname{\def\PY@tc##1{\textcolor[rgb]{0.10,0.09,0.49}{##1}}}
\expandafter\def\csname PY@tok@sx\endcsname{\def\PY@tc##1{\textcolor[rgb]{0.00,0.50,0.00}{##1}}}
\expandafter\def\csname PY@tok@m\endcsname{\def\PY@tc##1{\textcolor[rgb]{0.40,0.40,0.40}{##1}}}
\expandafter\def\csname PY@tok@gh\endcsname{\let\PY@bf=\textbf\def\PY@tc##1{\textcolor[rgb]{0.00,0.00,0.50}{##1}}}
\expandafter\def\csname PY@tok@gu\endcsname{\let\PY@bf=\textbf\def\PY@tc##1{\textcolor[rgb]{0.50,0.00,0.50}{##1}}}
\expandafter\def\csname PY@tok@gd\endcsname{\def\PY@tc##1{\textcolor[rgb]{0.63,0.00,0.00}{##1}}}
\expandafter\def\csname PY@tok@gi\endcsname{\def\PY@tc##1{\textcolor[rgb]{0.00,0.63,0.00}{##1}}}
\expandafter\def\csname PY@tok@gr\endcsname{\def\PY@tc##1{\textcolor[rgb]{1.00,0.00,0.00}{##1}}}
\expandafter\def\csname PY@tok@ge\endcsname{\let\PY@it=\textit}
\expandafter\def\csname PY@tok@gs\endcsname{\let\PY@bf=\textbf}
\expandafter\def\csname PY@tok@gp\endcsname{\let\PY@bf=\textbf\def\PY@tc##1{\textcolor[rgb]{0.00,0.00,0.50}{##1}}}
\expandafter\def\csname PY@tok@go\endcsname{\def\PY@tc##1{\textcolor[rgb]{0.53,0.53,0.53}{##1}}}
\expandafter\def\csname PY@tok@gt\endcsname{\def\PY@tc##1{\textcolor[rgb]{0.00,0.27,0.87}{##1}}}
\expandafter\def\csname PY@tok@err\endcsname{\def\PY@bc##1{\setlength{\fboxsep}{0pt}\fcolorbox[rgb]{1.00,0.00,0.00}{1,1,1}{\strut ##1}}}
\expandafter\def\csname PY@tok@kc\endcsname{\let\PY@bf=\textbf\def\PY@tc##1{\textcolor[rgb]{0.00,0.50,0.00}{##1}}}
\expandafter\def\csname PY@tok@kd\endcsname{\let\PY@bf=\textbf\def\PY@tc##1{\textcolor[rgb]{0.00,0.50,0.00}{##1}}}
\expandafter\def\csname PY@tok@kn\endcsname{\let\PY@bf=\textbf\def\PY@tc##1{\textcolor[rgb]{0.00,0.50,0.00}{##1}}}
\expandafter\def\csname PY@tok@kr\endcsname{\let\PY@bf=\textbf\def\PY@tc##1{\textcolor[rgb]{0.00,0.50,0.00}{##1}}}
\expandafter\def\csname PY@tok@bp\endcsname{\def\PY@tc##1{\textcolor[rgb]{0.00,0.50,0.00}{##1}}}
\expandafter\def\csname PY@tok@fm\endcsname{\def\PY@tc##1{\textcolor[rgb]{0.00,0.00,1.00}{##1}}}
\expandafter\def\csname PY@tok@vc\endcsname{\def\PY@tc##1{\textcolor[rgb]{0.10,0.09,0.49}{##1}}}
\expandafter\def\csname PY@tok@vg\endcsname{\def\PY@tc##1{\textcolor[rgb]{0.10,0.09,0.49}{##1}}}
\expandafter\def\csname PY@tok@vi\endcsname{\def\PY@tc##1{\textcolor[rgb]{0.10,0.09,0.49}{##1}}}
\expandafter\def\csname PY@tok@vm\endcsname{\def\PY@tc##1{\textcolor[rgb]{0.10,0.09,0.49}{##1}}}
\expandafter\def\csname PY@tok@sa\endcsname{\def\PY@tc##1{\textcolor[rgb]{0.73,0.13,0.13}{##1}}}
\expandafter\def\csname PY@tok@sb\endcsname{\def\PY@tc##1{\textcolor[rgb]{0.73,0.13,0.13}{##1}}}
\expandafter\def\csname PY@tok@sc\endcsname{\def\PY@tc##1{\textcolor[rgb]{0.73,0.13,0.13}{##1}}}
\expandafter\def\csname PY@tok@dl\endcsname{\def\PY@tc##1{\textcolor[rgb]{0.73,0.13,0.13}{##1}}}
\expandafter\def\csname PY@tok@s2\endcsname{\def\PY@tc##1{\textcolor[rgb]{0.73,0.13,0.13}{##1}}}
\expandafter\def\csname PY@tok@sh\endcsname{\def\PY@tc##1{\textcolor[rgb]{0.73,0.13,0.13}{##1}}}
\expandafter\def\csname PY@tok@s1\endcsname{\def\PY@tc##1{\textcolor[rgb]{0.73,0.13,0.13}{##1}}}
\expandafter\def\csname PY@tok@mb\endcsname{\def\PY@tc##1{\textcolor[rgb]{0.40,0.40,0.40}{##1}}}
\expandafter\def\csname PY@tok@mf\endcsname{\def\PY@tc##1{\textcolor[rgb]{0.40,0.40,0.40}{##1}}}
\expandafter\def\csname PY@tok@mh\endcsname{\def\PY@tc##1{\textcolor[rgb]{0.40,0.40,0.40}{##1}}}
\expandafter\def\csname PY@tok@mi\endcsname{\def\PY@tc##1{\textcolor[rgb]{0.40,0.40,0.40}{##1}}}
\expandafter\def\csname PY@tok@il\endcsname{\def\PY@tc##1{\textcolor[rgb]{0.40,0.40,0.40}{##1}}}
\expandafter\def\csname PY@tok@mo\endcsname{\def\PY@tc##1{\textcolor[rgb]{0.40,0.40,0.40}{##1}}}
\expandafter\def\csname PY@tok@ch\endcsname{\let\PY@it=\textit\def\PY@tc##1{\textcolor[rgb]{0.25,0.50,0.50}{##1}}}
\expandafter\def\csname PY@tok@cm\endcsname{\let\PY@it=\textit\def\PY@tc##1{\textcolor[rgb]{0.25,0.50,0.50}{##1}}}
\expandafter\def\csname PY@tok@cpf\endcsname{\let\PY@it=\textit\def\PY@tc##1{\textcolor[rgb]{0.25,0.50,0.50}{##1}}}
\expandafter\def\csname PY@tok@c1\endcsname{\let\PY@it=\textit\def\PY@tc##1{\textcolor[rgb]{0.25,0.50,0.50}{##1}}}
\expandafter\def\csname PY@tok@cs\endcsname{\let\PY@it=\textit\def\PY@tc##1{\textcolor[rgb]{0.25,0.50,0.50}{##1}}}

\def\PYZbs{\char`\\}
\def\PYZus{\char`\_}
\def\PYZob{\char`\{}
\def\PYZcb{\char`\}}
\def\PYZca{\char`\^}
\def\PYZam{\char`\&}
\def\PYZlt{\char`\<}
\def\PYZgt{\char`\>}
\def\PYZsh{\char`\#}
\def\PYZpc{\char`\%}
\def\PYZdl{\char`\$}
\def\PYZhy{\char`\-}
\def\PYZsq{\char`\'}
\def\PYZdq{\char`\"}
\def\PYZti{\char`\~}
% for compatibility with earlier versions
\def\PYZat{@}
\def\PYZlb{[}
\def\PYZrb{]}
\makeatother


    % Exact colors from NB
    \definecolor{incolor}{rgb}{0.0, 0.0, 0.5}
    \definecolor{outcolor}{rgb}{0.545, 0.0, 0.0}



    
    % Prevent overflowing lines due to hard-to-break entities
    \sloppy 
    % Setup hyperref package
    \hypersetup{
      breaklinks=true,  % so long urls are correctly broken across lines
      colorlinks=true,
      urlcolor=urlcolor,
      linkcolor=linkcolor,
      citecolor=citecolor,
      }
    % Slightly bigger margins than the latex defaults
    
    \geometry{verbose,tmargin=1in,bmargin=1in,lmargin=1in,rmargin=1in}
    
    

    \begin{document}
    
    
    \maketitle
    
    

    
    \subsubsection*{Q1}\label{q1}

    \begin{enumerate}
\def\labelenumi{\arabic{enumi}.}
\tightlist
\item
  Use R to randomly assign 10 experimental units to each of three
  treatments (1, 2, and 3). Then simulate responses for the 30
  experimental units satisfying the one-way ANOVA model:
\end{enumerate}

\[Y_{it} = \mu + \tau_i + \epsilon_{it}, i=1,2,...,v, t=1,2,,..., r_i\]

\[\epsilon_{it} \sim N(0, \sigma^2)\]

with \(\mu=4.7, \sigma^2=4\), and treatment effects
\(\tau_1 = -3, \tau_2 = 5, \tau_3=-2\). Your solution should include
your R code and a plot of the simulated values.

    \begin{Verbatim}[commandchars=\\\{\}]
{\color{incolor}In [{\color{incolor}40}]:} treatments.not.random\PY{o}{=}\PY{k+kt}{c}\PY{p}{(}\PY{k+kp}{rep}\PY{p}{(}\PY{l+s}{\PYZdq{}}\PY{l+s}{1\PYZdq{}}\PY{p}{,}\PY{l+m}{10}\PY{p}{)}\PY{p}{,} \PY{k+kp}{rep}\PY{p}{(}\PY{l+s}{\PYZdq{}}\PY{l+s}{2\PYZdq{}}\PY{p}{,}\PY{l+m}{10}\PY{p}{)}\PY{p}{,} \PY{k+kp}{rep}\PY{p}{(}\PY{l+s}{\PYZdq{}}\PY{l+s}{3\PYZdq{}}\PY{p}{,} \PY{l+m}{10}\PY{p}{)}\PY{p}{)}
         treatment\PY{o}{=}\PY{k+kp}{sample}\PY{p}{(}treatments.not.random\PY{p}{)}
         experiment\PYZus{}unit \PY{o}{=} \PY{l+m}{1}\PY{o}{:}\PY{k+kp}{length}\PY{p}{(}treatment\PY{p}{)}
         table\PY{o}{=}\PY{k+kt}{data.frame}\PY{p}{(}experiment\PYZus{}unit\PY{p}{,} treatment\PY{p}{,}row.names\PY{o}{=}\PY{k+kc}{NULL}\PY{p}{)}
         \PY{k+kp}{table}
\end{Verbatim}

    \begin{tabular}{r|ll}
 experiment\_unit & treatment\\
\hline
	  1 & 1 \\
	  2 & 3 \\
	  3 & 2 \\
	  4 & 2 \\
	  5 & 2 \\
	  6 & 3 \\
	  7 & 3 \\
	  8 & 1 \\
	  9 & 1 \\
	 10 & 1 \\
	 11 & 2 \\
	 12 & 1 \\
	 13 & 3 \\
	 14 & 3 \\
	 15 & 2 \\
	 16 & 2 \\
	 17 & 3 \\
	 18 & 2 \\
	 19 & 3 \\
	 20 & 2 \\
	 21 & 1 \\
	 22 & 3 \\
	 23 & 3 \\
	 24 & 1 \\
	 25 & 2 \\
	 26 & 1 \\
	 27 & 3 \\
	 28 & 1 \\
	 29 & 1 \\
	 30 & 2 \\
\end{tabular}


    
    \begin{Verbatim}[commandchars=\\\{\}]
{\color{incolor}In [{\color{incolor}41}]:} mu\PY{o}{=}\PY{l+m}{4.7}
         tau\PYZus{}1\PY{o}{=}\PY{l+m}{\PYZhy{}3}
         tau\PYZus{}2\PY{o}{=}\PY{l+m}{5}
         tau\PYZus{}3\PY{o}{=}\PY{l+m}{\PYZhy{}2}
         var\PY{o}{=}\PY{l+m}{4}
         
         means\PYZus{}q1\PY{o}{=}\PY{k+kp}{rep}\PY{p}{(}\PY{k+kc}{NA}\PY{p}{,}\PY{k+kp}{length}\PY{p}{(}treatment\PY{p}{)}\PY{p}{)}
         means\PYZus{}q1\PY{p}{[}treatment\PY{o}{==}\PY{l+m}{1}\PY{p}{]} \PY{o}{=} mu\PY{o}{+}tau\PYZus{}1
         means\PYZus{}q1\PY{p}{[}treatment\PY{o}{==}\PY{l+m}{2}\PY{p}{]} \PY{o}{=} mu\PY{o}{+}tau\PYZus{}2
         means\PYZus{}q1\PY{p}{[}treatment\PY{o}{==}\PY{l+m}{3}\PY{p}{]} \PY{o}{=} mu\PY{o}{+}tau\PYZus{}3
         
         y\PYZus{}simulate\PYZus{}q1 \PY{o}{=} means\PYZus{}q1\PY{o}{+}rnorm\PY{p}{(}n \PY{o}{=} \PY{k+kp}{length}\PY{p}{(}means\PYZus{}q1\PY{p}{)}\PY{p}{,} mean \PY{o}{=} \PY{l+m}{0}\PY{p}{,} sd \PY{o}{=} \PY{k+kp}{sqrt}\PY{p}{(}var\PY{p}{)}\PY{p}{)}
         sim\PYZus{}data \PY{o}{=} \PY{k+kt}{data.frame}\PY{p}{(}experiment\PYZus{}unit\PY{p}{,} treatment\PY{p}{,} y\PYZus{}simulate\PYZus{}q1\PY{p}{)}
         sim\PYZus{}data
\end{Verbatim}

    \begin{tabular}{r|lll}
 experiment\_unit & treatment & y\_simulate\_q1\\
\hline
	  1         & 1          & -1.6184361\\
	  2         & 3          &  5.8802735\\
	  3         & 2          &  6.3770675\\
	  4         & 2          & 14.3735955\\
	  5         & 2          &  8.4595723\\
	  6         & 3          &  3.8001343\\
	  7         & 3          &  1.0086134\\
	  8         & 1          &  0.3531091\\
	  9         & 1          &  1.0026467\\
	 10         & 1          &  2.3112544\\
	 11         & 2          & 10.4067991\\
	 12         & 1          &  3.3594491\\
	 13         & 3          &  4.4810127\\
	 14         & 3          &  0.9192064\\
	 15         & 2          & 12.2877769\\
	 16         & 2          &  6.5982482\\
	 17         & 3          &  0.2219408\\
	 18         & 2          & 14.4265322\\
	 19         & 3          &  2.1988443\\
	 20         & 2          &  8.8380465\\
	 21         & 1          &  5.7932315\\
	 22         & 3          &  1.3083453\\
	 23         & 3          &  3.1476776\\
	 24         & 1          & -0.3273547\\
	 25         & 2          &  8.9700277\\
	 26         & 1          & -0.9235407\\
	 27         & 3          &  0.8706975\\
	 28         & 1          &  2.6829802\\
	 29         & 1          &  2.7401941\\
	 30         & 2          &  9.1540193\\
\end{tabular}


    
    \begin{Verbatim}[commandchars=\\\{\}]
{\color{incolor}In [{\color{incolor}42}]:} \PY{k+kn}{library}\PY{p}{(}ggplot2\PY{p}{)}
         p1\PY{o}{\PYZlt{}\PYZhy{}}ggplot\PY{p}{(}sim\PYZus{}data\PY{p}{,} aes\PY{p}{(}x\PY{o}{=}treatment\PY{p}{,} y\PY{o}{=}y\PYZus{}simulate\PYZus{}q1\PY{p}{,} color\PY{o}{=}treatment\PY{p}{)}\PY{p}{)} \PY{o}{+}
           geom\PYZus{}boxplot\PY{p}{(}\PY{p}{)} \PY{o}{+}
           ylab\PY{p}{(}\PY{l+s}{\PYZsq{}}\PY{l+s}{simulated y\PYZsq{}}\PY{p}{)} \PY{o}{+} 
           ggtitle\PY{p}{(}\PY{l+s}{\PYZsq{}}\PY{l+s}{Boxplots of simulated responses\PYZsq{}}\PY{p}{)} \PY{o}{+}
           theme\PY{p}{(}plot.title \PY{o}{=} element\PYZus{}text\PY{p}{(}hjust \PY{o}{=} \PY{l+m}{0.5}\PY{p}{)}\PY{p}{)} 
         p1
\end{Verbatim}

    
    
    \begin{center}
    \adjustimage{max size={0.5\linewidth}{0.5\paperheight}}{output_4_1.png}
    \end{center}
    { \hspace*{\fill} \\}
    
    \subsubsection*{Q2}\label{q2}

    Consider the situation in Problem 1. The experimenter wants to consider
a reduced model where \(\tau 1 = \tau 2 = \tau 3 = 0\). Simulate
responses for the 30 experimental units satisfying this reduced model.
Compare boxplots of simulated responses under this reduced model with
boxplots of simulated responses under the full model described in
Problem 1 (where there are differences in the treatment effects).

    \begin{Verbatim}[commandchars=\\\{\}]
{\color{incolor}In [{\color{incolor}48}]:} tau\PYZus{}1\PY{o}{=}\PY{l+m}{0}
         tau\PYZus{}2\PY{o}{=}\PY{l+m}{0}
         tau\PYZus{}3\PY{o}{=}\PY{l+m}{0}
         
         means\PYZus{}q2\PY{o}{=}\PY{k+kp}{rep}\PY{p}{(}\PY{k+kc}{NA}\PY{p}{,}\PY{k+kp}{length}\PY{p}{(}treatment\PY{p}{)}\PY{p}{)}
         means\PYZus{}q2\PY{p}{[}treatment\PY{o}{==}\PY{l+m}{1}\PY{p}{]} \PY{o}{=} mu\PY{o}{+}tau\PYZus{}1
         means\PYZus{}q2\PY{p}{[}treatment\PY{o}{==}\PY{l+m}{2}\PY{p}{]} \PY{o}{=} mu\PY{o}{+}tau\PYZus{}2
         means\PYZus{}q2\PY{p}{[}treatment\PY{o}{==}\PY{l+m}{3}\PY{p}{]} \PY{o}{=} mu\PY{o}{+}tau\PYZus{}3
         
         y\PYZus{}simulate\PYZus{}q2 \PY{o}{=} means\PYZus{}q2\PY{o}{+}rnorm\PY{p}{(}n \PY{o}{=} \PY{k+kp}{length}\PY{p}{(}means\PYZus{}q2\PY{p}{)}\PY{p}{,} mean \PY{o}{=} \PY{l+m}{0}\PY{p}{,} sd \PY{o}{=} \PY{k+kp}{sqrt}\PY{p}{(}var\PY{p}{)}\PY{p}{)}
         sim\PYZus{}data \PY{o}{=} \PY{k+kt}{data.frame}\PY{p}{(}experiment\PYZus{}unit\PY{p}{,} treatment\PY{p}{,} y\PYZus{}simulate\PYZus{}q2\PY{p}{)}
         sim\PYZus{}data
\end{Verbatim}

    \begin{tabular}{r|lll}
 experiment\_unit & treatment & y\_simulate\_q2\\
\hline
	  1         & 1          &  4.5871138\\
	  2         & 3          &  8.9409421\\
	  3         & 2          &  5.4448399\\
	  4         & 2          & -0.2890012\\
	  5         & 2          &  9.4563168\\
	  6         & 3          &  5.6918686\\
	  7         & 3          & -8.2416622\\
	  8         & 1          &  7.3036280\\
	  9         & 1          &  8.7815548\\
	 10         & 1          &  5.0297383\\
	 11         & 2          & -0.2905075\\
	 12         & 1          &  8.4399276\\
	 13         & 3          &  3.1789920\\
	 14         & 3          &  2.0923404\\
	 15         & 2          &  5.4524241\\
	 16         & 2          &  9.3299666\\
	 17         & 3          &  1.6604748\\
	 18         & 2          &  7.1431085\\
	 19         & 3          &  4.3337423\\
	 20         & 2          &  7.0824088\\
	 21         & 1          &  3.3082532\\
	 22         & 3          &  3.0837190\\
	 23         & 3          &  9.4843263\\
	 24         & 1          &  7.7620453\\
	 25         & 2          &  2.4644266\\
	 26         & 1          &  2.0500897\\
	 27         & 3          &  8.6995005\\
	 28         & 1          &  9.1221858\\
	 29         & 1          &  9.4040856\\
	 30         & 2          &  7.8276827\\
\end{tabular}


    
    \begin{Verbatim}[commandchars=\\\{\}]
{\color{incolor}In [{\color{incolor}49}]:} p2\PY{o}{\PYZlt{}\PYZhy{}}ggplot\PY{p}{(}sim\PYZus{}data\PY{p}{,} aes\PY{p}{(}x\PY{o}{=}treatment\PY{p}{,} y\PY{o}{=}y\PYZus{}simulate\PYZus{}q2\PY{p}{,} color\PY{o}{=}treatment\PY{p}{)}\PY{p}{)} \PY{o}{+}
           geom\PYZus{}boxplot\PY{p}{(}\PY{p}{)} \PY{o}{+}
           ylab\PY{p}{(}\PY{l+s}{\PYZsq{}}\PY{l+s}{simulated y\PYZsq{}}\PY{p}{)} \PY{o}{+} 
           ggtitle\PY{p}{(}\PY{l+s}{\PYZsq{}}\PY{l+s}{Boxplots of simulated responses\PYZsq{}}\PY{p}{)} \PY{o}{+}
           theme\PY{p}{(}plot.title \PY{o}{=} element\PYZus{}text\PY{p}{(}hjust \PY{o}{=} \PY{l+m}{0.5}\PY{p}{)}\PY{p}{)} 
         p2
\end{Verbatim}

    
    
    \begin{center}
    \adjustimage{max size={0.5\linewidth}{0.5\paperheight}}{output_8_1.png}
    \end{center}
    { \hspace*{\fill} \\}
    
    \begin{Verbatim}[commandchars=\\\{\}]
{\color{incolor}In [{\color{incolor}50}]:} \PY{k+kn}{library}\PY{p}{(}gridExtra\PY{p}{)}
         grid.arrange\PY{p}{(}p1\PY{p}{,} p2\PY{p}{,} nrow \PY{o}{=} \PY{l+m}{2}\PY{p}{)}
\end{Verbatim}

    \begin{center}
    \adjustimage{max size={0.7\linewidth}{0.7\paperheight}}{output_9_0.png}
    \end{center}
    { \hspace*{\fill} \\}
    
    As we can see from the two boxplots above, it's apparent that the means
of three treatments of the reduced model are more similar to each other,
since:

\[Y_{1.} = \mu + \tau_1 + \epsilon = \mu + \epsilon_{1.}\]
\[Y_{2.} = \mu + \tau_2 + \epsilon = \mu + \epsilon_{2.}\]
\[Y_{3.} = \mu + \tau_3 + \epsilon = \mu + \epsilon_{3.}\]

Other than the normally distributed noise, the mean \(\mu\) of all three
treatments are basically the same.

    \subsubsection*{Q3}\label{q3}

    3.Now explore what happens to data simulated from the model in Problem 1
when the error variance increases. Try multiple values for \(\sigma^2\)
and find a value of \(\sigma^2\) for which you cannot see any noticable
difference in the boxplots of response values from the three treatments.

    \begin{Verbatim}[commandchars=\\\{\}]
{\color{incolor}In [{\color{incolor}47}]:} vars \PY{o}{=} \PY{l+m}{1}\PY{o}{:}\PY{l+m}{12}
         boxplots \PY{o}{\PYZlt{}\PYZhy{}} \PY{k+kt}{list}\PY{p}{(}\PY{p}{)}
         \PY{k+kr}{for} \PY{p}{(}var \PY{k+kr}{in} vars\PY{p}{)}\PY{p}{\PYZob{}}
             y\PYZus{}simulate \PY{o}{=} means\PYZus{}q1\PY{o}{+}rnorm\PY{p}{(}n \PY{o}{=} \PY{k+kp}{length}\PY{p}{(}means\PYZus{}q1\PY{p}{)}\PY{p}{,} mean \PY{o}{=} \PY{l+m}{0}\PY{p}{,} sd \PY{o}{=} \PY{k+kp}{sqrt}\PY{p}{(}var\PY{p}{)}\PY{p}{)}
             sim\PYZus{}data \PY{o}{=} \PY{k+kt}{data.frame}\PY{p}{(}experiment\PYZus{}unit\PY{p}{,} treatment\PY{p}{,} y\PYZus{}simulate\PY{p}{)}
             p\PY{o}{\PYZlt{}\PYZhy{}}ggplot\PY{p}{(}sim\PYZus{}data\PY{p}{,} aes\PY{p}{(}x\PY{o}{=}treatment\PY{p}{,} y\PY{o}{=}y\PYZus{}simulate\PY{p}{,} color\PY{o}{=}treatment\PY{p}{)}\PY{p}{)} \PY{o}{+}
                       geom\PYZus{}boxplot\PY{p}{(}\PY{p}{)} \PY{o}{+}
                       ylab\PY{p}{(}\PY{l+s}{\PYZsq{}}\PY{l+s}{simulated y\PYZsq{}}\PY{p}{)} \PY{o}{+} 
                       ggtitle\PY{p}{(}\PY{k+kp}{paste}\PY{p}{(}\PY{l+s}{\PYZsq{}}\PY{l+s}{sigma\PYZca{}2=\PYZsq{}}\PY{p}{,} var\PY{p}{)}\PY{p}{)} \PY{o}{+}
                       theme\PY{p}{(}plot.title \PY{o}{=} element\PYZus{}text\PY{p}{(}hjust \PY{o}{=} \PY{l+m}{0.5}\PY{p}{)}\PY{p}{)}
             boxplots\PY{p}{[[}var\PY{p}{]]} \PY{o}{\PYZlt{}\PYZhy{}} p
             \PY{p}{\PYZcb{}}
         grid.arrange\PY{p}{(}grobs \PY{o}{=} boxplots\PY{p}{,} nrow\PY{o}{=}\PY{l+m}{4}\PY{p}{)}
\end{Verbatim}

    \begin{center}
    \adjustimage{max size={0.9\linewidth}{0.9\paperheight}}{output_13_0.png}
    \end{center}
    { \hspace*{\fill} \\}
    
    from the boxplots above we could see that when \(\sigma^2 \ge 10\), the
boxplots of response values from the three treatments seems similar
compared to each other.

    \subsubsection*{Q4}\label{q4}

    Under the model in Problem 1, what is the distribution of \(Y_{23}\) ,
the response from the 3rd experimental unit to receive treatment 2?

\[Y_{23} = \mu + \tau_2 + \epsilon_{23} = 4.7 + 5 + \epsilon_{23} = 9.7 + \epsilon_{23}\]

\[\epsilon_{23} \sim N(0, 4)\]

\[\therefore Y_{23} \sim N(9.7, 4)\]

    \subsubsection*{Q5}\label{q5}

    Under the model in Problem 1, what is the distribution of

\[\bar Y_2 = \frac{1}{r_2} \sum^{r_2}_{t=1} Y_{2t}\]

    \begin{center}\rule{1.1\linewidth}{\linethickness}\end{center}

Since

\[Y_{2t} = \mu + \tau_2 + \epsilon_{2t} = 4.7 + 5 + \epsilon_{2t} = 9.7 + \epsilon_{2t}\]

\[\epsilon_{2t} \sim N(0, 4)\]

\[\therefore Y_{2t} \sim N(9.7, 4)\]

\[\bar Y_2 = \frac{1}{10} \sum^{10}_{t=1} Y_{2t} \sim N(9.7\cdot 10 \cdot \frac{1}{10}, 4 \cdot 10 \cdot (\frac{1}{10})^2)\]

\[\therefore \bar Y_2 \sim N(9.7, 0.4)\]

    \subsubsection*{Q6}\label{q6}
    
    Under the model in Problem 1, what is the distribution of the difference between an experimental unit
    receiving treatment 1 and an experimental unit receiving treatment 2\\

    Since

\[Y_{1t} = \mu + \tau_1 + \epsilon_{1t} = 4.7 - 3 + \epsilon_{1t} = 1.7 + \epsilon_{1t}\]

\[\epsilon_{1t} \sim N(0, 4)\]

\[\therefore Y_{1t} \sim N(1.7, 4)\]

\[Y_{2t} = \mu + \tau_2 + \epsilon_{2t} = 4.7 + 5 + \epsilon_{2t} = 9.7 + \epsilon_{2t}\]

\[\epsilon_{2t} \sim N(0, 4)\]

\[\therefore Y_{2t} \sim N(9.7, 4)\]

Hence

\[E \left [Y_{1t} - Y_{2t} \right ] = E \left [ Y_{1t} \right ] - E \left [ Y_{2t} \right ] = 1.7 - 9.7 = -8\]

\[Var \left (Y_{1t} - Y_{2t} \right ) = Var \left (Y_{1t} \right ) + Var \left (Y_{2t} \right ) = 4+4 = 8\]

\[\therefore Y_{1t} - Y_{2t} \sim N(-8, 8)\]


    % Add a bibliography block to the postdoc
    
    
    
    \end{document}
